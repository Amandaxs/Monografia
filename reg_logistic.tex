%%%%%%%%%%%%%%%%%%%%%%%%%%%%%%%%%%%%%%%%%%%%%%%%%%%%%%%%%%%%%%%%%%%%%%%%%
%%%%%%%%%%%%%%%%%%%%%%%%%%% Regressão logística %%%%%%%%%%%%%%%%%%%%%%%%%
%%%%%%%%%%%%%%%%%%%%%%%%%%%%%%%%%%%%%%%%%%%%%%%%%%%%%%%%%%%%%%%%%%%%%%%%%

Umas das técnicas mais conhecidas, quando estamos falando em classificação binária é a regressão logística.

Seja Y nossa variável aleatória definida por:

\enspace

$
Y = 
\begin{cases}

             1 & \mbox{se o devedor quitou pelo menos 80\% da dívida}\\

             0 & \mbox{caso contrário.}

       \end{cases}
$

\hspace{2cm}

A relação entre a probabilidade de sucesso \bm{$p_i$} e as variáveis explicativas será dada através da função de ligação logística definida por:
%%%%%
$ logit(p_i) = log\{\frac{p_i}{1 - p_i}\} = \beta_0 + \beta_1 x_{1i} + ... + \beta_p x_{pi} $
